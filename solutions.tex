\documentclass[11pt]{article}
\usepackage[top=1cm,bottom=2cm,left=1.5cm,right=1.5cm]{geometry}
\usepackage{hyperref}

\usepackage{amsmath}
\usepackage{mathrsfs}
\usepackage{mathtools}

\usepackage{algpseudocode}
\usepackage{algorithm}
\usepackage{array}

\usepackage{graphics}
\usepackage{float}
\usepackage{caption}
% don't ident paragraphs
\usepackage{parskip}
\usepackage{subfig}
%\usepackage{gfsdidot}
\usepackage{xltxtra}
\usepackage{xgreek} \setmainfont[Mapping=tex-text]{GFS Didot}
\setmonofont[Mapping=tex-text]{Fira Mono}
\usepackage{amssymb}
\usepackage{multicol}
\usepackage{parcolumns}
\usepackage{pifont}



\newcommand{\cmark}{\ding{51}}%
\newcommand{\xmark}{\ding{55}}%
\newcommand*{\vertbar}{\rule[-1ex]{0.5pt}{2.5ex}}
\newcommand*{\horzbar}{\rule[.5ex]{9.5ex}{0.5pt}}
\newcommand*{\horzbarr}{\rule[.5ex]{8.0ex}{0.5pt}}

\usepackage{listings, color}
\definecolor{green}{rgb}{0,0.5,0}
\definecolor{gray}{rgb}{0.4,0.4,0.4}
\definecolor{lblue}{rgb}{0.9,0.9,1}
\definecolor{red}{rgb}{0.9,0,0}
\definecolor{blue}{rgb}{0.6,0,0.6}

\lstset{
  backgroundcolor=\color{lblue},
  basicstyle=\footnotesize\ttfamily,
  breaklines=true,
  columns=fullflexible,
  commentstyle=\color{gray},
  keywordstyle=\color{blue},
  stringstyle=\color{red},
  identifierstyle=\color{black},
  %numbers=left,
  %numberstyle=\scriptsize,
  %numbersep=5pt,
  showstringspaces=false
}

\linespread{1.4} %(changes linespacing. Default 1, one and a half 1.3, double 1.6)

%-------------------------------------------------------------------------------------------
%Here begins the document!
\begin{document}

\noindent\begin{minipage}{\textwidth}

  %A
  \begin{minipage}[b]{.6\textwidth}
    \large{
      Μεταπτυχιακό ΑΛΜΑ\\
      Ζαχαράκης Αλέξανδρος \href{mailto:azacharakis@yandex.com}{azacharakis@yandex.com}\\
      Τσιάρας Λάμπρος \href{mailto:std08262@di.uoa.gr}{std08262@di.uoa.gr} \\
      Ακαδημαικό Έτος 2016-2017
    }
    \\
  \end{minipage}

\end{minipage}
\\ \\ \\
\begin{center}
  \textbf{\Large{Θεωρία Αναδρομής}}\\
  \textbf{\Large{1η Σειρά Ασκήσεων}}
\end{center}                


% ------------------------------------- Άσκηση 2 -------------------------------------------
\section*{Άσκηση 2}
Θα δείξουμε ότι το πρόβλημα είναι αποφάνσιμο κατασκευάζοντας μηχανή Turing που το αποφασίζει.
Παρατηρούμε το εξής: οι διαμορφώσεις μιας μηχανής turing $M=(Q,\Sigma,\Gamma,q_0,q_{yes},q_{no})$ 
που η κεφαλή δεν περνάει από τη θέση $n$ της ταινίας είναι πεπερασμένες. Συγκεκριμένα κάθε διαμόρφωση μπορεί
να κωδικοποιηθεί ως $(q,i,w)$ όπου $q$ είναι η κατάσταση της μηχανής, $i$ η θέση της κεφαλής και $w$ η 
λέξη που περιέχει η ταινία.  Αφού μετράμε τις καταστάσεις που η κεφαλή δεν φτάνει στη θέση $n$ το
μήκος της $w$ είναι μικρότερο από $n$.  Άρα όλες οι πιθανές διαφορετικές διαμορφώσεις της $M$ είναι το 
πολύ $C=|Q|\cdot n\cdot |\Sigma|^n$. Αν λοιπόν τρέξουμε τη μηχανή $M$ για περισσότερα από $C$ βήματα
έχουμε τρεις περιπτώσεις: 
\begin{enumerate}
    \item Θα περάσει η κεφαλή από τη θέση $n$
    \item Θα τερματίσει η μηχανή χωρίς να περάσει η κεφαλή από τη θέση $n$
    \item Μία διαμόρφωση θα επαναληφθεί.
\end{enumerate}
Αν συμβεί το τρίτο ενδεχόμενο ο υπολογισμός δεν θα τερματίσει ποτέ αφού η Μηχανή θα επαναλαμβάνει 
τα ίδια configuration (ο υπολογισμός είναι ντετερμινιστικός) και η κεφαλή δεν θα περάσει ποτέ από 
τη θέση $n$ αν δεν έχει ήδη περάσει.

Θα χρησιμοποιήσουμε την καθολική μηχανή Turing $U$ για να προσομοιώσουμε $C+1$ βήματα της $M$. 
Η μηχανή που θα κάνει την προσομοίωση θα έχει δύο επιπλέον ταινίες από αυτές που χρειάζεται 
η $U$. Αρχικά στην μία επιπλέον ταινία εκτελούμε $n$ κινήσεις δεξιά και βάζουμε ένα ειδικό 
σύμβολο $*$ σε αυτή τη θέση. Στην άλλη κρατάμε έναν δυαδικό μετρητή ξεκινόντας από $0$. 

Η μηχανή $U$ προσομοιώνει ένα ένα τα βήματα της $Μ$. Σε κάθε βήμα που προσωμοιώνει κάνει τις εξής
επιπλέον δουλειές: Κινεί την 1η επιπλέον κεφαλή ακριβώς όπως κινείται η κεφαλή της $M$, αυξάνει 
κατά 1 τον μετρητή, ελέγχει αν διάβασε το ειδικό σύμβολο $*$ και αν ναι πηγαίνει στην κατάσταση αποδοχής 
και τέλος συγκρίνει τον μετρητή με την ποσότητα $C$ και αν είναι μεγαλύτερος από αυτή πηγαίνει 
στην κατάσταση απόρριψης.
                                                              

% ------------------------------------------------------------------------------------------



% ------------------------------------- Άσκηση 4 -------------------------------------------
\section*{Άσκηση 4}
Θα κατασκευάσουμε την αναδρομική γλώσσα $C$ ώς τη γλώσσα που αποφασίζει μια μηχανή Turing 
(δηλαδή θα κατασκευάσουμε $M_C$ και θα ορίσουμε $C=L(M_C)$. Η μηχανή $M_C$
χρησιμοποιεί τις $M_{coA}$ και $M_{coB}$ που ημιαποφασίζουν τις $\overline{A},\overline{B}\in RE$.
Με είσοδο $w$ H $M_C$ προσομοιώνει "παράλληλα" (ένα βήμα την καθεμία) τις $M_{coA}$, $M_{coB}$
ξεκινώντας από την $M_{coA}$. Αν η $M_{coA}$ αποδεχτεί τότε η $M_C$ αποδέχεται. Αν η $M_{coB}$ αποδεχτεί
τότε η $M_C$ απορρίπτει. Ισχύουν τα κάτωθι:
\begin{itemize}
  \item  $\overline{A}\cup \overline{B} = \Sigma^*$ γιατί αν δεν ίσχυε τότε για κάποιο $x$ θα είχαμε ότι
    $x\notin\overline{A}\cup \overline{B} \Rightarrow x\in\overline{\overline{A}\cup \overline{B}}=A\cap B$
    που είναι άτοπο από υπόθεση
  \item $C\in REC$ γιατί αφού η ένωση είναι το $\Sigma^*$ όπως δείξαμε η παραπάνω μηχανή τερματίζει σε 
    κάθε πιθανή είσοδο είτε σε αποδοχή είτε σε απόρριψη.
  \item $B\subseteq C$. Γιατί αν $x\in B$ η $M_{coB}$ δεν θα τερματίσει αλλά θα τερματίσει η  
    $M_{coA}$ και θα αποδεχτεί οπότε η $M_C$ αποδέχεται εκ κατασκευής
  \item $A\subseteq \overline{C}$. Γιατί αν $x\in A$ η $M_{coA}$ δεν θα τερματίσει αλλά θα τερματίσει η  
    $M_{coB}$ και θα αποδεχτεί οπότε η $M_C$ απορρίπτει εκ κατασκευής
\end{itemize}
Τελικά όλα τα παραπάνω δείχνουν ότι η $C$ διαχωρίζει τις $A,B$.

% ------------------------------------------------------------------------------------------

% ------------------------------------- Άσκηση 5 -------------------------------------------
\section*{Άσκηση 5}
Έστω \textit{L} μία άπειρη αναδρομικά απαριθμήσιμη γλώσσα. Θα υπάρχει απαριθμητής $E_L$ που θα απαριθμεί τα στοιχεία της γλώσσας (σε τυχαία σειρά). Θα αντιστοιχίσουμε τις λέξεις που προκύπτουν από τα άρτια βήματα εκτέλεσης στη γλώσσα $L_1$ και τις λέξεις που προκύπτουν από τα περιττά βήματα στη γλώσσα $L_2$, ελέγχοντας τυχόν προηγούμενες εμφανίσεις της λέξης μέχρι εκείνο το βήμα (μπορεί να γίνει με μία μηχανή Turing που γράφει σε μία ταινία τις λέξεις μέχρι το σημείο \textit{n} που είμαστε και ελέγχει αν υπάρχει εκεί η λέξη). Αν η λέξη έχει ξαναεμφανιστεί την παραλείπουμε και προχωράμε. Έτσι η γλώσσα $L_1$ θα περιλαμβάνει όλες τις (μοναδικές) λέξεις που προέκυψαν από τον απαριθμητή για άρτια βήματα, η γλώσσα $L_2$ αντίστοιχα όλες τις (μοναδικές) λέξεις που προέκυψαν από τον απαριθμητή για περιττά βήματα και ισχύει ότι $L_1 \cap L_2 = \emptyset$ και $L_1 \cup L_2 = L$.


%-------------------------------------------------------------------------------------------


% ------------------------------------- Άσκηση 7 -------------------------------------------
\section*{Άσκηση 7}
\begin{itemize}
  \item \textbf{ΛΑΘΟΣ} Θα δείξουμε ότι υπάρχει $x\in R$ τέτοιο ώστε $x\notin K$. Έστω
    $\phi_x$ η συνάρτηση που δεν ορίζεται πουθενά δηλαδή $(\forall n)\phi_x(n)=\uparrow$. To 
    $DOM(\phi_x)$ είναι αναδρομικό αφού είναι το κενό το οποίο αναγνωρίζεται από την μηχανή
    Turing που για κάθε είσοδο απορρίπτει. Εξ' ορισμού $\phi_x(x)=\uparrow$ και $M_x(x)=\uparrow$
    δηλαδή $x\notin K$.
  \item \textbf{ΛΑΘΟΣ} Θα δείξουμε ότι υπάρχει $x$ που ανήκει και στις δύο. Με αντίστοιχο με πριν
    τρόπο επιλέγουμε $x$ τον αριθμό Goedel της σταθερής ολικής συνάρτησης $\phi(x)=42$. Το πεδίο
    ορισμού της είναι αναδρομικό (παίρνουμε την Μ.T. που αποδέχεται κάθε λέξη) και επίσης $M_x(x)=\downarrow$
    αφού η μηχανή για κάθε είσοδο τυπώνει $42$. Συνεπώς $x\in M\cap K$ 
  \item ?
  \item \textbf{ΛΑΘΟΣ} Υπάρχει ένα προς ένα και επί $\sigma:K\rightarrow\overline{K}$. Έστω $M_1,M_2,\ldots$
    η αρίθμηση των μηχανών. Έστω $f_1:\mathbb{N}\rightarrow \mathbb{N}$ που απεικονίζει $n\mapsto i$ όπου $i$ είναι 
    ο αριθμός Goedel της n-οστής Μηχανής που τερματίζει και $f_2:\mathbb{N}\rightarrow \mathbb{N}$ που απεικονίζει
    $n\mapsto i$ όπου $i$ είναι ο αριθμός Goedel της n-οστής Μηχανής που δεν τερματίζει. Οι συναρτήσεις είναι καλά
    ορισμένες αφού και στις δύο κάθε στοιχείο αντιστοιχίζεται μονοσήμαντα στην εικόνα (δεν γίνεται μια θέση να 
    καταλαμβάνεται από δύο μηχανές αφού έχουμε διάταξη)
    και είναι ολικές αφού για κάθε $n$ υπάρχει σίγουρα η $n$-οστή μηχανή που (δεν)
    τερματίζει αλλιώς θα υπήρχε μέγιστο στοιχείο και το ($\overline{K}$) $K$ θα ήταν πεπερασμένο. 
    Η $\sigma$ απεικονίζει $x\mapsto y$ τέτοια ώστε αν $f_1(n)=x$ τότε $f_2(n)=y$. Είναι επί γιατί για κάθε 
    $y\in\overline{K}$ υπάρχει προεικόνα της $f_2$, έστω $n$ (κάπου είναι στη σειρά των μηχανών που δεν τερματίζουν με είσοδο τον
    εαυτό τους) και υπάρχει $x$ τέτοιο ώστε $f_1(x) = n$. Είναι 1-1 γιατί αν $\sigma(x)=\sigma(x')=y$ τότε για κάποια $n,n'$ 
    $f_1(n)=x$ και $f_1(n')=x'$ και $y=f_2(n)=f_2(n')$ που συνεπάγεται ότι $n=n'$ και $x=x'$.
\end{itemize}

% ------------------------------------------------------------------------------------------


% ------------------------------------- Άσκηση 8 -------------------------------------------
\section*{Άσκηση 8}
Θα κατασκευάσουμε μηχανή Turing που να την υπολογίζει. Με είσοδο $i$ η $M$ θα χρησιμοποιεί ως υπορουτίνα τη
μηχανή $M_i$ ως εξής. Θα αρχικοποιεί έναν μετρητή στο 0 και σε κάθε γύρο θα προσομοιώνει κάποια βήματα
της $M_i$ για διαφορετικές εισόδους. Θα κρατάει επίσης και έναν άλλον μετρητή $c$ που θα κρατάει το πλήθος 
των στοιχείων που αποδέχεται η γλώσσα. Στον γύρο 1 θα τρέχει την $M_i$ με είσοδο $1$ για 1 βήμα, στο 
γύρο 2 θα τρέχει τη $M_i$ με εισόδους $1,2$ για δύο βήματα και γενικά στο γύρο $j$ θα τρέχει τη $M_i$ για $j$ 
βήματα στις εισόδους $1,2,\ldots,j$. Στον $j$ γύρο, αν η μηχανή $M_i$ με είσοδο $j$ τερματίσει ακριβώς στο $j$-οστό
βήμα (και όχι νωρίτερα για να μην διπλομετράμε) τότε αυξάνουμε κατά $1$ τον μετρητή $c$ και αν κάποια στιγμή ο 
$c$ γίνει μεγαλύτερος του $i$ τυπώνει $i$ και τερματίζει. Αν $|L(M_i)|\geq i$ τότε σε πεπερασμένο χρόνο η μηχανή 
θα τερματίσει με σωστό αποτέλεσμα, ενώ αντίθετα θα κολλήσει όπως πρέπει αφού η $\phi$ εκεί δεν ορίζεται.

% ------------------------------------------------------------------------------------------



% ------------------------------------- Άσκηση 10 -------------------------------------------
\section*{Άσκηση 10}
Θα κατασκευάσουμε μηχανή Turing $M_f$ που να την υπολογίζει. Θεωρούμε ότι $M_i$ υπολογίζει την $\phi_i$.
Με είσοδο $n$ η μηχανή δουλεύει ως εξής: Σε μία ταινία κρατάει το μερικό άθροισμα ξεκινώντας από 0 και σε μία
άλλη ταινία αρχικοποιεί έναν μετρητή σε 0 που μετράει επαναλήψεις. Στην i-οστή επανάληψη προσομοιώνει τη λειτουργία
της $M_i$ με είσοδο $n$ και το αποτέλεσμα που παίρνει (αν φυσικά τερματίζει) το προσθέτει στο άθροισμα και πηγαίνει στον
επόμενο γύρο. Το αποτέλεσμα είναι σωστό αφού αν $\phi_1(n)\downarrow,\ldots,\phi_n(n)\downarrow$ η μηχανή θα τερματίσει
με το άθροισμα, αλλιώς θα κολλήσει όπως θα έπρεπε αφού σε αυτή τη περίπτωση η $f$ δεν ορίζεται.

% ------------------------------------------------------------------------------------------




% ------------------------------------- Άσκηση 11 -------------------------------------------
\section*{Άσκηση 11}
$(\Rightarrow)$ Έστω $L$ μη κενή αναδρομικά απαριθμήσιμη γλώσσα. Τότε υπάρχει απαριθμητής $E$ 
που απαριθμεί τα στοιχεία της. Ορίζουμε $f$ ως εξής: $f(n) = \text{το ι-οστό στοιχείο που τυπώνει ο }E$. 
\begin{itemize}
  \item Η $f$ είναι υπολογίσιμη. Θα κατασκευάσουμε μηχανή Turing $M_f$ που υπολογίζει την $f$. Η $M_f$ 
    λειτουργεί ως εξής: Η μηχανή παίρνει σαν είσοδο $n$. Ξεκινάει και τρέχει τον $E$ και κάθε φορά 
    που αυτός πηγαίνει στην κατάσταση εκτύπωσης η $M_f$ αυξάνει έναν μετρητή και τον συγκρίνει με το $n$.
    Αν οι δύο τιμές είναι ίσες σταματάει τυπώνοντας το στοιχείο που έδωσε ο $E$. 
  \item Η $f$ είναι επί. Από τον ορισμό ισχύει ότι κάθε στοιχείο $x\in L$ κάποια στιγμή θα τυπωθεί
    από τον απαριθμητή και συνεπώς κάθε τέτοιο στοιχείο έχει προεικόνα.
\end{itemize}

$(\Leftarrow)$ Έστω ότι υπάρχει $f:\mathbb{N}\rightarrow L$ που να είναι υπολογίσιμη και επί. Θα
κατασκευάσουμε μηχανή Turing $M_L$ που να ημιαποφασίζει την $L$. Έστω $M_f$ η μηχανή που υπολογίζει 
την $f$. H $M_L$ δουλεύει ως εξής. Με είσοδο $w$ προσομοιώνει ένα βήμα της εκτέλεσης $M_f(1)$. Στη συνέχεια 
προσομοιώνει $2$ βήματα της εκτέλεσης $M_f(1)$ και $M_f(2)$ κ.ο.κ. (δηλαδή σε κάθε "γύρο" $t$ προσομοιώνει
$t$ βήματα των εκτελέσεων $M_f(i)$ για $i\in\{1,2,\ldots,t\}$. Όταν κάποια από τις εκτελέσεις $M_f(i)$ τελειώσει ελέγχει
αν η έξοδός της εκτέλεσης είναι ίση με $w$. Αν είναι αποδέχεται αλλιώς συνεχίζει να εκτελείται κανονικά.
Επειδή η $f$ είναι επί κάθε στοιχείο της $x\in L$ έχει προεικόνα δηλαδή για κάποιο $n\in \mathbb{N}$ θα ισχύει ότι
$f(n)=x$ και συνεπώς η μηχανή για κάθε στοιχείο της γλώσσας σταματάει αποδεχόμενη.

% ------------------------------------------------------------------------------------------



% ------------------------------------- Άσκηση 12 -------------------------------------------
\section*{Άσκηση 12}
Διακρίνουμε δύο περιπτώσεις. Αν $Dom(f)=\emptyset$ τότε $\forall n\in \mathbb{N}\;\; f(n) = \uparrow$ και 
συνεπώς έχουμε ότι $Im(f)=\emptyset$ το οποίο σύνολο είναι αναδρομικό και αναγνωρίζεται από την 
μηχανή Turing με $q_{\text{αρχική}} = q_{no}$.

Αν $Dom(f)\neq\emptyset$ τότε ως μη κενό υποσύνολο των φυσικών το $Dom(f)$ έχει ελάχιστο στοιχείο, έστω $n$ 
και έστω $f(n) = m$. Η $f$ λοιπόν δεν ορίζεται για στοιχεία μικρότερα του $n$ και επειδή είναι και φθίνουσα
το $m$ είναι το μέγιστο στοιχείο του $Im(f)$. Άρα $Im(f) \subseteq \{1,2,\ldots,m\}$ και συνεπώς είναι 
πεπερασμένο. Όμως κάθε πεπερασμένο σύνολο είναι αναδρομικό.

Για το δεύτερο κομμάτι θεωρούμε ότι $L\subseteq\mathbb{N}$ για να βγαίνει νόημα (το οποίο είναι OK αφού  
$\Sigma*$ και $\mathbb{N}$ είναι ισομορφικά). Η απόδειξη του ζητούμενου είναι ακριβώς ίδια με παραπάνω, 
δηλαδή είτε η εικόνα των στοιχείων του $L$ είναι το κενό σύνολο, είτε έχει όπως και πριν ελάχιστο στοιχείο.

% ------------------------------------------------------------------------------------------




\end{document}
