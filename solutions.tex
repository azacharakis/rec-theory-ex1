\documentclass[11pt]{article}
\usepackage[top=1cm,bottom=2cm,left=1.5cm,right=1.5cm]{geometry}
\usepackage{hyperref}

\usepackage{amsmath}
\usepackage{mathrsfs}
\usepackage{mathtools}

\usepackage{algpseudocode}
\usepackage{algorithm}
\usepackage{array}

\usepackage{graphics}
\usepackage{float}
\usepackage{caption}
% don't ident paragraphs
\usepackage{parskip}
\usepackage{subfig}
\usepackage{gfsdidot}
\usepackage{xltxtra}
\usepackage{xgreek} \setmainfont[Mapping=tex-text]{GFS Didot}
\setmonofont[Mapping=tex-text]{Fira Mono}
\usepackage{amssymb}
\usepackage{multicol}
\usepackage{parcolumns}
\usepackage{pifont}



\newcommand{\cmark}{\ding{51}}%
\newcommand{\xmark}{\ding{55}}%
\newcommand*{\vertbar}{\rule[-1ex]{0.5pt}{2.5ex}}
\newcommand*{\horzbar}{\rule[.5ex]{9.5ex}{0.5pt}}
\newcommand*{\horzbarr}{\rule[.5ex]{8.0ex}{0.5pt}}

\usepackage{listings, color}
\definecolor{green}{rgb}{0,0.5,0}
\definecolor{gray}{rgb}{0.4,0.4,0.4}
\definecolor{lblue}{rgb}{0.9,0.9,1}
\definecolor{red}{rgb}{0.9,0,0}
\definecolor{blue}{rgb}{0.6,0,0.6}

\lstset{
  backgroundcolor=\color{lblue},
  basicstyle=\footnotesize\ttfamily,
  breaklines=true,
  columns=fullflexible,
  commentstyle=\color{gray},
  keywordstyle=\color{blue},
  stringstyle=\color{red},
  identifierstyle=\color{black},
  %numbers=left,
  %numberstyle=\scriptsize,
  %numbersep=5pt,
  showstringspaces=false
}

\linespread{1.4} %(changes linespacing. Default 1, one and a half 1.3, double 1.6)

%-------------------------------------------------------------------------------------------
%Here begins the document!
\begin{document}

\noindent\begin{minipage}{\textwidth}

  %A
  \begin{minipage}[b]{.6\textwidth}
    \large{
      Μεταπτυχιακό ΑΛΜΑ\\
      Ζαχαράκης Αλέξανδρος \href{mailto:azacharakis@yandex.com}{azacharakis@yandex.com}\\
      Τσιάρας Λάμπρος \href{mailto:std08262@di.uoa.gr}{std08262@di.uoa.gr} \\
      Ακαδημαικό Έτος 2016-2017
    }
    \\
  \end{minipage}

\end{minipage}
\\ \\ \\
\begin{center}
  \textbf{\Large{Θεωρία Αναδρομής}}\\
  \textbf{\Large{1η Σειρά Ασκήσεων}}
\end{center}                


% ------------------------------------- Άσκηση 2 -------------------------------------------
\section*{Άσκηση 2}
Θα δείξουμε ότι το πρόβλημα είναι αποφάνσιμο κατασκευάζοντας μηχανή Turing που το αποφασίζει.
Παρατηρούμε το εξής: οι διαμορφώσεις μιας μηχανής turing $M=(Q,\Sigma,\Gamma,q_0,q_{yes},q_{no})$ 
που η κεφαλή δεν περνάει από τη θέση $n$ της ταινίας είναι πεπερασμένες. Συγκεκριμένα κάθε διαμόρφοση μπορεί
να κωδικοποιηθεί ως $(q,i,w)$ όπου $q$ είναι η κατάσταση της μηχανής, $i$ η θέση της κεφαλής και $w$ η 
λέξη που περιέχει η ταινία.  Αφού μετράμε τις καταστάσεις που η κεφαλή δεν φτάνει στη θέση $n$ το
μήκος της $w$ είναι μικρότερο από $n$.  Άρα όλες οι πιθανές διαφορετικές διαμορφώσεις της $M$ είναι το 
πολύ $C=|Q|\cdot n\cdot |\Sigma|^n$. Αν λοιπόν τρέξουμε τη μηχανή $M$ για περισσότερα από $C$ βήματα
έχουμε τρεις περιπτώσεις: 
\begin{enumerate}
    \item Θα περάσει η κεφαλή από τη θέση $n$
    \item Θα τερματίσει η μηχανή χωρίς να περάσει η κεφαλή από τη θέση $n$
    \item Μία διαμόρφωση θα επαναληφθεί.
\end{enumerate}
Αν συμβεί το τρίτο ενδεχόμενο ο υπολογισμός δεν θα τερματίσει ποτέ αφού η Μηχανή θα επαναλαμβάνει 
τα ίδια configuration (ο υπολογισμός είναι ντετερμινιστικός) και η κεφαλή δεν θα περάσει ποτέ από 
τη θέση $n$ αν δεν έχει ήδη περάσει.

Θα χρησιμοποιήσουμε την καθολική μηχανή Turing $U$ για να προσωμοιώσουμε $C+1$ βήματα της $M$. 
Η μηχανή που θα κάνει την προσωμοίωση θα έχει δύο επιπλέον ταινίες από αυτές που χρειάζεται 
η $U$. Αρχικά στην μία επιπλέον ταινία εκτελοούμε $n$ κινήσεις δεξιά και βάζουμε ένα ειδικό 
σύμβολο $*$ σε αυτή τη θέση. Στην άλλη κρατάμε έναν δυαδικό μετρητή ξεκινόντας από $0$. 

Η μηχανή $U$ προσωμοιώνει ένα ενα τα βήματα της $Μ$. Σε κάθε βήμα που προσωμοιώνει κάνει τις εξής
επιπλέον δουλειές: Κινεί την 1η επιπλέον κεφαλή ακριβώς όπως κινήται η κεφαλή της $M$, αυξάνει 
κατά 1 τον μετρητή, ελέγχει αν διάβασε το ειδικό σύμβολο $*$ και αν ναι πηγαίνει στην κατάσταση αποδοχής 
και τέλος συγκρίνει τον μετρητή με την ποσότητα $C$ και αν είναι μεγαλύτερος από αυτή πηγαίνει 
στην κατάσταση απόρριψης.
                                                              

% ------------------------------------------------------------------------------------------




% ------------------------------------- Άσκηση 12 -------------------------------------------
\section*{Άσκηση 12}
Διακρίνουμε δύο περιπτώσεις. Αν $Dom(f)=\emptyset$ τότε $\forall n\in \mathbb{N}\;\; f(n) = \uparrow$ και 
συνεπώς έχουμε ότι $Im(f)=\emptyset$ το οποίο σύνολο είναι αναδρομικό και αναγνωρίζεται από την 
μηχανή Turing με $q_{\text{αρχική}} = q_{no}$.

Αν $Dom(f)\neq\emptyset$ τότε ως μη κενό υποσύνολο των φυσικών το $Dom(f)$ έχει ελάχιστο στοιχείο, έστω $n$ 
και έστω $f(n) = m$. Η $f$ λοιπόν δεν ορίζεται για στοιχεία μικρότερα του $n$ και επειδή είναι και φθίνουσα
το $m$ είναι το μέγιστο στοιχείο του $Im(f)$. Αρα $Im(f) \subseteq \{1,2,\ldots,m\}$ και συνεπώς είναι 
πεπερασμένο. Όμως κάθε πεπερασμένο σύνολο είναι αναδρομικό.

Για το δεύτερο κομμάτι θεωρούμε ότι $L\subseteq\mathbb{N}$ για να βγαίνει νόημα (το οποίο είναι OK αφού  
$\Sigma*$ και $\mathbb{N}$ είναι ισομορφικά). Η απόδειξη του ζητούμενου είναι ακριβώς ίδια με παραπάνω, 
δηλαδή είτε η εικόνα των στοιχείων του $L$ είναι το κενό σύνολο, είτε έχει όπως και πριν ελάχιστο στοιχείο.

% ------------------------------------------------------------------------------------------




\end{document}
