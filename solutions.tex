\documentclass[11pt]{article}
\usepackage[top=1cm,bottom=2cm,left=1.5cm,right=1.5cm]{geometry}
\usepackage{hyperref}

\usepackage{amsmath}
\usepackage{mathrsfs}
\usepackage{mathtools}

\usepackage{algpseudocode}
\usepackage{algorithm}
\usepackage{array}

\usepackage{graphics}
\usepackage{float}
\usepackage{caption}
% don't ident paragraphs
\usepackage{parskip}
\usepackage{subfig}
\usepackage{gfsdidot}
\usepackage{xltxtra}
\usepackage{xgreek} \setmainfont[Mapping=tex-text]{GFS Didot}
\setmonofont[Mapping=tex-text]{Fira Mono}
\usepackage{amssymb}
\usepackage{multicol}
\usepackage{parcolumns}
\usepackage{pifont}



\newcommand{\cmark}{\ding{51}}%
\newcommand{\xmark}{\ding{55}}%
\newcommand*{\vertbar}{\rule[-1ex]{0.5pt}{2.5ex}}
\newcommand*{\horzbar}{\rule[.5ex]{9.5ex}{0.5pt}}
\newcommand*{\horzbarr}{\rule[.5ex]{8.0ex}{0.5pt}}

\usepackage{listings, color}
\definecolor{green}{rgb}{0,0.5,0}
\definecolor{gray}{rgb}{0.4,0.4,0.4}
\definecolor{lblue}{rgb}{0.9,0.9,1}
\definecolor{red}{rgb}{0.9,0,0}
\definecolor{blue}{rgb}{0.6,0,0.6}

\lstset{
  backgroundcolor=\color{lblue},
  basicstyle=\footnotesize\ttfamily,
  breaklines=true,
  columns=fullflexible,
  commentstyle=\color{gray},
  keywordstyle=\color{blue},
  stringstyle=\color{red},
  identifierstyle=\color{black},
  %numbers=left,
  %numberstyle=\scriptsize,
  %numbersep=5pt,
  showstringspaces=false
}

\linespread{1.4} %(changes linespacing. Default 1, one and a half 1.3, double 1.6)

%-------------------------------------------------------------------------------------------
%Here begins the document!
\begin{document}

\noindent\begin{minipage}{\textwidth}

  %A
  \begin{minipage}[b]{.6\textwidth}
    \large{
      Μεταπτυχιακό ΑΛΜΑ\\
      Ζαχαράκης Αλέξανδρος \href{mailto:azacharakis@yandex.com}{azacharakis@yandex.com}\\
      Τσιάρας Λάμπρος \href{mailto:std08262@di.uoa.gr}{std08262@di.uoa.gr} \\
      Ακαδημαικό Έτος 2016-2017
    }
    \\
  \end{minipage}

\end{minipage}
\\ \\ \\
\begin{center}
  \textbf{\Large{Θεωρία Αναδρομής}}\\
  \textbf{\Large{1η Σειρά Ασκήσεων}}
\end{center}                


% ------------------------------------- Άσκηση 2 -------------------------------------------
\section*{Άσκηση 2}



% ------------------------------------------------------------------------------------------

\end{document}
